\documentclass[11pt,a4paper]{article}

% basic packages
\usepackage{float}
\usepackage{polski}
\usepackage{amsmath}
\usepackage{graphicx}
\usepackage{parskip}
\usepackage[utf8x]{inputenc}
\usepackage{fullpage}

% bibliography and links
\usepackage{url}
\usepackage{cite}
\def\UrlBreaks{\do\/\do-}
\usepackage[hidelinks]{hyperref}

% graphs
\usepackage{tikz}
\usetikzlibrary{arrows}

\begin{document}

\begin{titlepage}
  \begin{center}

    \textsc{\Large Politechnika Warszawska}\\[0.1cm]
    \small Wydział Elektroniki i Technik Informacyjnych
    \vfill

    \textsc{\small Inteligentne Systemy Informacyjne}\\[0.1cm]
    \Huge Eliza-like Chatting Bot\\[1.5cm]

    \vfill

    \begin{minipage}{0.4\textwidth}
      \begin{flushleft} \large
        \emph{Autor:}\\[0.1cm]
        Jacek \textsc{Witkowski}\\
      \end{flushleft}
    \end{minipage}

    \vfill
    {\large 6 lutego 2015}

  \end{center}
\end{titlepage}
\section{Wstęp}
Zadaniem projektu było stworzenie programu potrafiącego prowadzić rozmowę
z~użytkownikiem. Przyjąłem, że~temat rozmowy będzie ograniczony do~muzyki
rockowej, ponieważ wcześniej posiadałem już bazę reguł mogącą posłużyć
za~fundament budowy odpowiedniej bazy wiedzy dla~tego projektu.

\section{Opis działania}
Program swoje działanie opiera na bazie reguł, które zawierają następujące
informacje:
\begin{enumerate}
  \item zbiór możliwych wejść (tekstów wpisywanych przez użytkownika),
  \item zbior możliwych wyjść (komunikatów wypisywanych przez program),
  \item (opcjonalnie) informacja czy reguła ma zakończyć działanie programu.
\end{enumerate}

Przykładowo, gdy użytkownik wpisze ,,WHAT IS YOUR NAME'' otrzyma odpowiedź
,,MY~NAME IS~ELIZA''.

\subsection{Przebieg interakcji z użytkownikiem}
Konwersację zawsze rozpoczyna użytkownik. Tekst wpisany przez użytkownika jest
przekazywany do~obiektu \textit{ResponsesGenerator}, którego zadaniem jest
wygenerowanie odpowiedzi. W~pierwszej kolejności obiekt sprawdza czy~użytkownik
w poprzednim komunikacie nie napisał tego samego co bieżącym. Jeśli tak, to
wysyłana jest jedna z odpowiedzi ze zbioru \textit{repeatResponseDescriptors}.
Jeśli nie~to:
\begin{enumerate}
  \item wyszukiwane są reguły, których wejście pasuje do wpisanego przez
  użytkownika (czyli te, których wejście zawiera się w początku napisu
  podanego prez użytkownika),
  \item wyizolowanie tych reguł, które najlepiej pasują do~tekstu wprowadzonego
  przez~użytkownika,
  \item z~najbardziej dopasowanych reguł, wybierana jest losowo jedna,
  \item tworzona jest odpowiedź na podstawie wybranej reguły i wpisanego przez
  użytkownika tekstu
\end{enumerate}

\subsection{Przetwarzanie odpowiedzi}
W czwartym kroku algorytmu przedstawionego wyżej generowana jest odpowiedź
na~podstawie reguły i wejścia. Jeśli w~regule wyjście jest napisem
nie~zawierającym znakow *, to jest ono po prostu zwracane w oryginalnej formie.
Jeśli natomiast znaki * występują, wówczas są one zastępowane tekstem
generowanym na~podstawie wejścia użytkownika. Przykładowo załóżmy, że użytkownik
wpisał \textit{,,MY NAME IS JACK''}, a istnieje reguła o wejściu: \textit{,,MY
NAME IS''} i wyjściu \textit{DO YOU LIKE MUSIC,*?}. Wówczas * zostanie
zastąpiona tekstem, który ,,wystaje'' poza to, co~udało się~dopasować.
Wiadomością, którą otrzyma użytkownik, będzie: \textit{DO YOU LIKE MUSIC,
\textbf{JACK}?}.

\subsubsection{Transpozycje} Dodatkowym krokiem podczas przetwarzania odpowiedzi
jest zastosowanie transpozycji, czyli zamiana niektórych słów zgodnie ze
zdefiniowanymi regułami (np. ,,I'M'' $\rightarrow$ ,,YOU'RE''). Transpozycji
poddawane są tylko te teksty, które są wstawiane zamiast * w odpowiedzi.
Przykładowo załóżmy, że reguła ma wejście: \textit{,,I THINK I''} i~wyjście
\textit{,,DO YOU REALLY*?''}. Wówczas jeśli użytkownik wpisze \textit{,,I THINK
I LIKE YOU''} otrzyma odpowiedź: \textit{,,DO YOU REALLY LIKE ME?''}.

\section{Możliwości rozwoju}
W~pierwszym kroku należało by ulepszyć utrzymywanie kontekstu wypowiedzi
(obecnie pamiętana jest co najwyżej jedna poprzednia odpowiedź, co~jest sporym
uproszczeniem). Należałoby stworzyć ontologię reprezentującą różne tematy
rozmów i~ich~wzajemne relacje. Wówczas reguły mogłby by definiować w~jakim kontekście
mają być stosowane i~do jakiego kontekstu należy przejść.

Drugim możliwym usprawnieniem byłoby wprowadzenie mechanizmu samouczenia.
W najprostszej wersji program mógłby przyjmować bezpośrednio
od~użytkownika sugestie jak~należy odpowiadać na zadane pytania.

W kolejnym kroku można byłoby ulepszyć dopasowywanie reguł do~wprowadzonego
przez~użytkownika napisu. Przykładowo można byłoby wyszukiwać odpowiednie
reguły wg~słów kluczowych (wymagałoby to sprowadzenia każdego słowa
występującego w~tekście wprowadzonym przez~użytkownika do~swojej podstawowej
formy).
\end{document}